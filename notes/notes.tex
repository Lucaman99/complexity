\documentclass[superscriptaddress,pra,nofootinbib,notitlepage, floatfix]{revtex4-1}
\usepackage{amssymb}
\usepackage{dsfont}
\usepackage{complexity}
\usepackage[utf8]{inputenc} % allow utf-8 input
\usepackage[T1]{fontenc}    % use 8-bit T1 fonts
\usepackage{hyperref}       % hyperlinks
\usepackage{url}            % simple URL typesetting
\usepackage{booktabs}       % professional-quality tables
\usepackage{amsfonts}       % blackboard math symbols
\usepackage{nicefrac}       % compact symbols for 1/2, etc.
\usepackage{microtype}      % microtypography
\usepackage{lipsum}
\usepackage{amsmath}
\usepackage{physics}
\usepackage{placeins}
%\newcommand{\ketbra}[2]{\ket{#1}\!\bra{#2}}
\usepackage{bbold}
\usepackage{tikz}
\usepackage{forest}
\usepackage{multirow}
\usetikzlibrary{calc,arrows.meta,positioning, shapes.geometric, arrows, shapes.symbols,shapes.callouts,patterns}

\tikzset{
    every node/.style={font=\sffamily\small},
    main node/.style={thick,circle ,draw},
    visible node/.style={thick,rectangle ,draw}
}
%\usepackage[dvipsnames]{xcolor}
% \usepackage{subfig}
%\usepackage{subcaption}
\usepackage{qcircuit}
\usepackage{graphicx}
\usepackage{graphics}
\usepackage{amsthm}
%\usepackage{subcaption}
\usepackage[caption=false]{subfig}
\usepackage{latexsym}
\usepackage{floatrow}
\usepackage{mathtools}
\usepackage{verbatim}
\usepackage{upgreek}
\usepackage{textgreek}
\usepackage{thmtools}
\usepackage{thm-restate}

\usepackage{hyperref}

\usepackage{cleveref}

\newtheorem{theorem}{Theorem}
\newtheorem{corollary}[theorem]{Corollary}
\newtheorem{proposition}[theorem]{Proposition}
\newtheorem{lemma}[theorem]{Lemma}
\newcommand{\vect}[1]{\boldsymbol{#1}}
\DeclareMathOperator{\EX}{\mathbb{E}}
\newcommand{\Var}{\mathrm{Var}}
\declaretheorem[name=Lemma,numberwithin=section]{lm}

\begin{document}

\title{Qubit Complexity Notes}

\date{\today}

\maketitle

\section{Automatic Computation of the Metric Tensor and Christoffel Symbols}

In order to determine the complexity of some unitary $U$, denoted $C[U]$, we must determine the length of the geodesic extending from $I$ to $U$ with respect to some metric on the space of unitaries.
In general, a metric on the space of unitaries is given by:

\begin{equation}
  ds^2 = \text{Tr}(i dU U^{\dagger} T_I) \mathcal{I}_{IJ} \text{Tr}(i dU U^{\dagger} T_J)
\end{equation}

where the $T_I$ are the generators of $SU(2^{N})$, for some $N$-qubit system. Given some parametrized unitary $U(\boldsymbol{\theta})$, we can write this metric explicity in terms of derivatives with respect to the parameters, as
$dU = \partial_{\boldsymbol{\theta}} U \ d\boldsymbol{\theta}$.
\newline

The case considered in the paper [CITE] considers a two-qubit system, with a parametrization as follows:

\begin{equation}
U(\boldsymbol{\theta}) = \exp( -i H(\boldsymbol\theta)) = \exp \left( -i [ \theta_x X + \theta_y Y + \theta_z Z ] \right)
\end{equation}


\bibliographystyle{unsrt}
\bibliography{ref}


\end{document}
